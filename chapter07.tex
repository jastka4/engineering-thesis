\chapter{Summary}
\section{Degree of Achievement}
The main objective of the thesis was to design and develop a system that students can use to access data in their student services system. It was supposed to be comfortable to use and quick to access. The vital thing was that it should work on both mobile operating systems, iOS, and Android. The transformation server configuration had to be easily editable and expendable and should not require any restarts. All of the previously stated objectives have been achieved, and the system is almost ready for publication in one of the digital distribution services.

Not all functional and non-functional requirements have been implemented. For example, users cannot change the language of the application since the only available language, as of now, is English. Additionally, students cannot calculate their average grade. Other requirements have been met.

\section{Further Development}
As mentioned before, the core of the application has been completed, and after minor changes, it can be published in the App Store or Google Play. Before all this happens, some universities will have to agree to share the API of their student services system. If this happens, all transformations would have to be defined and stored in the system. Luckily, the architecture of the server provides an effortless way to implement the required changes. The thing is that the main service is, in a sense, a microservice which allows for easy extensions if the design of the service is in any way incompatible with the given API.

\section{Conclusion}
