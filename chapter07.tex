\chapter{Summary}
\section{Degree of Achievement}
The main objective of the thesis was to design and develop a system that students could use to access data in their student services system. It was supposed to be comfortable to use and quick to access. The vital thing was that it should work on both mobile operating systems, iOS, and Android. It ought to have a local SQLite database where all received data could be stored and accessed offline if a need arises. The transformation server configuration had to be easily editable and expendable and should not require any restarts. It was supposed to connect two schema-incompatible systems and allow them to communicate with each other. The server should act as a middleman between the mobile application and external APIs. All of the previously stated objectives have been achieved, and the system is almost ready for publication in one of the digital distribution services.

Not all functional and non-functional requirements have been implemented. For example, users cannot change the language of the application since the only available language, as of now, is English. Additionally, students cannot calculate their average grade. Other requirements have been met.

\section{Further Development}
As mentioned earlier, the core of the application has been completed, and after minor changes, it can be published in the App Store or Google Play. Before all this takes place, some universities will have to agree to share the API of their student services system. If this happens, all transformations would have to be defined and stored in the system. Luckily, the architecture of the server provides an effortless way to implement the required changes. The main service is, in a sense, a microservice which allows for easy extensions if the service design is in any way incompatible with the given API. The application requires a considerable amount of testing before it can be published anywhere. It would need a trial run on a smartphone to see how long it takes to communicate with the external API through the transformation server.


\section{Conclusion}
Creating mobile applications using Flutter is very quick and effortless. It allows programmers to design beautiful user interfaces using Material Design. The documentation of the framework and the Dart language is extensive, simple to access, search, and read. Flutter gets more popular every day, which translates into a rapidly growing package base. The best thing about the framework is that it can create two (iOS and Android) native applications from one codebase. Each of them can be modified separately if any need arises. It was a real pleasure to use it for this project.

The transformation service required lots of designing and thinking. The best part of creating the whole system was to come up with the best available architecture for the designed tasks. Coding was just a small portion of the actual work. That shows the importance of the design phase throughout the entire application development cycle.

The project has been very educative and can be easily extended in the future. Let us hope that the development will continue so that it can be eventually released to the public. It can eventually help many students find the right class or access their grades.
