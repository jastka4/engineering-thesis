\chapter{System Deployment}
The mobile application can be installed from the attached APK files for Android devices. To do so, we need to copy a package for the chosen architecture onto a smartphone. Then we only have to click on the copied application and let it install on the device. Installing the iOS package is more complicated, so it is much better to run it in an XCode simulator instead of a mobile machine. We simply have to start the virtual device and drag and drop our package onto its screen. The Android application can also be run on an Android emulator.

In the case of this project, it is best to run it on a computer instead of a mobile device. We use MockServer to mimic the behavior of a real student services system API. The mock server is not embedded into the mobile application, and because of this, logging in and browsing screens is not possible for applications installed on smartphones.

The best way to test this proof of concept is to run the mobile application using an Android or iOS emulator and start the backend server using the steps described below.

Docker and Docker Compose were used to simplify the startup of the server application and the API mocking service. Both tools are required for the next steps, so they must be installed on the system. To start the whole backend, we need to run the following commands from the disk root directory:
\begin{lstlisting}[numbers=none]
cd server/release
docker-compose up
\end{lstlisting}

\noindent You can also run and stop Docker in the detached mode:
\begin{lstlisting}[numbers=none]
docker-compose up -d
docker-compose down
\end{lstlisting}