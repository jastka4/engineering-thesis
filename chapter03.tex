% TODO - explain briefly in the project assumptions chapter and move to the chapter about implementation?
\chapter{Tools and Technologies}
\section{Flutter}
Flutter is an open-source toolkit created by Google. It can be used to develop natively compiled applications for mobiles (iOS, Android), desktops (Mac, Linux, Windows, Google Fuchsia), and the web from a single codebase. Flutter's engine utilizes Google's Skia graphics library for low-level rendering. It is written primarily in C++. It also leverages platform-specific SDKs, for example, iOS and Android.

Developers use the Dart language when creating Flutter applications. The toolkit runs in the Dart virtual machine that features a just-in-time execution engine. It can be used while debugging an app to do a ``hot reload'', which modifies source files and then injects them into a running application. Flutter added a stateful hot reload on top of it, wherein most cases, changes to source code can be reflected immediately in the working app without requiring a restart or any loss of state.

Release versions of Flutter apps are compiled with ahead-of-time (AOT) compilation on both iOS and Android, making Flutter's high performance on mobile devices possible.~\cite{flutter-wiki}

Creating UIs in Flutter involves using composition to assemble widgets from other widgets. According to the docs, ``A widget is an immutable description of part of a user interface''.~\cite{flutter} They define what their view should look like, given their current state and configuration. The widget rebuilds its description when it's state changes because the framework diffs against the previous representation to determine the minimal changes needed in the underlying render tree to transition from one state to the next.

Flutter aims to provide 120 frames per second (FPS) performance on devices capable of 120Hz updates and 60 FPS otherwise.

Complex widgets can be built from other smaller ones. An app is actually the largest widget of all of them, often called ``MyApp''. Text component is a widget, but so is its TextStyle, than defines things like color, size, font-weight, and family. Some widgets represent things, others, like TextStyle, represent characteristics. There are also ones that do something, like StreamBuilder and FutureBuilder.

An alternative option to using widgets is to use the Foundation library's methods directly. They can be used to draw text, shapes, and imagery directly on the canvas. One of the frameworks that utilized this is the open-source Flame game engine.

\section{Dart}
Dart is a client-optimized programming language developed by Google. It is multiplatform and can be used to build desktop, mobile, backend, and web applications.

It is object-oriented, class defined, garbage-collected language. It uses a C-style syntax and can be transcompiled into JavaScript. It supports abstract classes, interfaces, mixins, static typing, reified generics, and a sound type system.

Dart code can be compiled ahead-of-time into machine code. Applications build with Flutter, a mobile app SDK built with Dart, are deployed to app stores as AOT compiled Dart code.

Dart version 2.6 was accompanied by a new extension dart2native. It allows composing a Dart program into self-contained executables on the macOS, Linux, and Windows desktop platforms. Earlier, this feature only exposed capability on iOS and Android mobile devices via Flutter.~\cite{dart-wiki}

\section{SQLite}
SQLite is a relational database management system contained in a C-language library. It is embedded into the end program, unlike many other database management systems. It is built into most computers, all mobile phones and is bundled inside lots of applications that people use every day. It is a popular selection as an embedded database software for client/local storage in application software such as web browsers. It is one of the most used database engines, as it is used today by several widespread browsers, embedded systems, such as mobile phones, operating systems, among others. SQLite has bindings to many programming languages.

SQLite is ACID-compliant and follows PostgreSQL syntax while implementing most of the SQL standard. It uses weakly and dynamically typed SQL syntax that does not guarantee the domain integrity. A string can be inserted into a column defined as an integer. SQLite will try to convert data between formats when appropriate, for example, the string ``1234'' into an integer. However, it does not guarantee such conversion and will store the data as-is if such conversion is not possible.~\cite{sqlite-wiki}

\section{Android Studio}
Android Studio is an integrated development environment created by Google and built on JetBrains' IntelliJ IDEA software. It was specially designed for Android development. It can be downloaded on macOS, Windows, and Linux based operating systems. Previous versions of Android Studio were based on Eclipse IDE.~\cite{android-studio}

Flutter apps can be built using any text editor combined with Flutter command-line tools. It is best to use the Flutter plugin with Android Studio. The plugin provides users with, for example, syntax highlighting, code completion, widget editing assist, run and debug support.

\section{Java}
\section{Spring Framework}
\subsection{Spring MVC}
\subsection{Spring Cloud Config}
\subsection{Spring Boot}
\section{Jolt}
Jolt (JsOn Language for Transform) is a transformation library written in Java. It allows developers to convert one JSON structure to another using a schema created in JSON. The tool provides a set of transformation types:

\begin{itemize}
    \item shift - copies data from input to the output tree;
    \item default - applies default values to the tree;
    \item remove - removes data from the tree;
    \item sort - sorts map keys alphabetically;
    \item cardinality - adjusts the cardinality of input data.
\end{itemize}

Each of the types has its DSL, which is called a specification, that defines the new structure for outgoing JSON data.

A basic approach for converting JSON to JSON in Java is to use XSLT or STX. The conversion sequence would look like this:
\begin{center}
\textbf{JSON -> XML -> XSLT/STX -> XML -> JSON}
\end{center}

With Jolt, the conversion sequence is simplified and looks like this:
\begin{center}
\textbf{JSON -> Specification JSON -> JSON}
\end{center}

The out-of-the-box Jolt should be able to do most of the structural transformation. Any complex transformation logic which cannot be expressed in standard terms can be plugged in via Java extension class with Jolt.~\cite{jolt}

\section{JSON}
\section{YAML}

\section{MockServer}
\section{Docker}
